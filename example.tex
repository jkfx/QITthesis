\documentclass{QITthesis}

\begin{document}

\fancyhf{}
\fancyfoot[C]{\zihao{-5}\tnr\thepage}
\renewcommand{\headrulewidth}{0pt}
\pagenumbering{Roman}
\begin{abstract}
    \zihao{-4}
    \setlength{\baselineskip}{20pt}
    将Wistar大鼠随机分为5组:正常对照组、模型对照组、皂苷低剂量组、皂苷中剂量组和皂苷高剂量组。饲料中添加1\%乳清酸建立大鼠脂肪肝模型,并同时分别给予不同剂量的海参皂苷(0.01\%、0.03\%、0.05\%),连续饲喂10d。分别测定大鼠血清总胆固醇(Total cholesterol,TC)、甘油三酯(Triglyceride,TG)、高密度脂蛋白胆固醇(High density lipoprotein-cholesterol,HDL-C)含量,大鼠肝脏TC、TG、磷脂(phospholipid,PL)含量,以及肝脏中脂肪酸合成酶(FAS)的活性。结果显示海参皂苷可降低血清和肝脏脂肪含量,对脂肪肝具有很好的预防作用。

    \vspace{\baselineskip}\bfseries\heiti 关键词:
    \normalfont\heiti 海参;皂苷;脂肪肝;乳清酸;脂肪代谢
\end{abstract}

\newpage
\ctexset{abstractname=\zihao{3}\tnr\bfseries\centering Abstract\vspace{.75\baselineskip}}
\begin{abstract}
    \zihao{-4}
    \tnr
    \setlength{\baselineskip}{20pt}
    Male Wistar rats were randomly divided into five groups, including normal group, control group, low, medium and high dose saponin group. Model rats were established by with 1\% dietary orotic acid. Saponin tested group were given diets incorporated with saponin at the levels of 0.01\%, 0.03\% and 0.05\%, respectively. After 10 days feeding, Serum total cholesterol (TC), triglyceride (TG), high density lipoprotein-cholesterol (HDL-c), hepatic lipid concentrations (TG, TC and phospholipids) and the activities of hepatic fatty acid synthase (FAS) were determined. The saponin of sea cucumber could improve the serum and hepatic lipids accumulation, and is beneficial to prevent fatty liver.

    \vspace{\baselineskip}\zihao{-4}\tnr\bfseries Keywords:
    \zihao{-4}\tnr\normalfont sea cucumber; saponin; fatty liver; orotic acid; lipid bolism
\end{abstract}
    
\end{document}